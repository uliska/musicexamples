%%%%%%%%%%%%%%%%%%%%%%%%%%%%%%%%%%%%%%%%%%%%%%%%%%%%%%%%%%%%%%%%%%%%%%%%%%%
%                                                                         %
%      This file is part of the 'openLilyLib' library.                    %
%                                ===========                              %
%                                                                         %
%              https://github.com/lilyglyphs/openLilyLib                  %
%                                                                         %
%  Copyright 2012-13 by Urs Liska, lilyglyphs@ursliska.de                 %
%                                                                         %
%  'openLilyLib' is free software: you can redistribute it and/or modify  %
%  it under the terms of the GNU General Public License as published by   %
%  the Free Software Foundation, either version 3 of the License, or      %
%  (at your option) any later version.                                    %
%                                                                         %
%  This program is distributed in the hope that it will be useful,        %
%  but WITHOUT ANY WARRANTY; without even the implied warranty of         %
%  MERCHANTABILITY or FITNESS FOR A PARTICULAR PURPOSE. See the           %
%  GNU General Public License for more details.                           %
%                                                                         %
%  You should have received a copy of the GNU General Public License      %
%  along with this program.  If not, see <http://www.gnu.org/licenses/>.  %
%                                                                         %
%%%%%%%%%%%%%%%%%%%%%%%%%%%%%%%%%%%%%%%%%%%%%%%%%%%%%%%%%%%%%%%%%%%%%%%%%%%

\documentclass{OLLbook}

\title{musicexamples \LaTeX{} package}
\author{Urs Liska}

\begin{document}
\maketitle
\begin{authorAbstract}{Urs Liska}
\texttt{musicexamples} is a set of tools intended for printing and managing music examples in \LaTeX{} documents.
It was developed with examples in mind that are produced using the LilyPond notation software%
\footnote{\url{www.lilypond.org}},
but it can also be used to handle any kind of images.

It consists of three parts: a \LaTeX{} package, a set of configuration files for LilyPond scores and a set of Python scripts (to be implemented).
\end{authorAbstract}

\vfill
\input{copyright-notice.inp}

\tableofcontents

\chapter{End User Documentation}
\section{Installation and Requirements}

\section{musicexamples.sty}

\texttt{musicexamples} is a package that defines environments and commands to handle music examples (scores and fragments) within \LaTeX{} documents.
It supports floating or non-floating examples, one- or multi-system examples and finally full-(one- or multi-)page scores to be inserted.
The examples are numbered in one list and can be output as one contigious list of music examples, regardless of their type of inclusion.

It was developed from the perspective of a user of the LilyPond notation software, but the package should work with any kind of image suitable for music examples.
The package has some parallels with LilyPond's \texttt{lilypond-book} script, but it doesn't understand itself as a competitor for this, but rather as a different approach for people with somewhat different needs.
For anybody writing (about) music it may also be a good idea to have a look at my \texttt{lilyglyphs} package that will eventually be merged into the openLilyLib family of resources.

\bigskip
In order to use \texttt{musicexamples} you simply write \cmd{usepackage\{musicexamples\}} or \cmd{RequirePackage\{musicexamples\}}.
You will then have access to the its commands and environments:

There are two environments to be used for music examples within a page: \texttt{musicexample} and \texttt{musicexampleNonFloat}.
The point of having a non-floating environment is \emph{not} to have more control over the placement of the item, but rather to allow it to cross page breaks, so that a group of music systems may flow over one or more pages. 
These environments do not print the music examples themselves but only provide the environment for them (as a \texttt{figure} environment doesn't already print the figure).
You use them like any other float environment, so you can optionally add the placement directive after the \cmd{begin} statement (of the floating version).
Inside the environment you add the contents (see below), a \cmd{label} and the \cmd{caption} to be used.
\begin{quote}
\begin{verbatim}
\begin{musicexample}[t]
  \includegraphics{exampleimage}
  \caption{A typical music example}
\end{musicexample}
\end{verbatim}
\end{quote}

This will print your image in a floating environment [preferrably at the top of a page], will take care of the numbering and prepares for the inclusion in a list of music examples.

The usage of \texttt{musicexampleNonFloat} is identical, except that it doesn't accept the optional placement argument.
It will print the example right where you inserted the environment, and while the floating version can only print the example in a box on a single page, this one can spread over page breaks -- provided the music systems are given as a series of images.

The captions are (to my knowledge) standard captions that can be influenced (formatted) with the commands of the \texttt{caption} package, the default layout and style being due to my needs when developing the package.
The caption label defaults to “Music Example”, the heading of the list to “List of Music Examples”.
In order to change them you can use the commands \cmd{setXmpCaptionLabel} and \cmd{setXmpListName} with one mandatory argument supplying the respective string.
\begin{quote}
\begin{verbatim}
\setXmpCaptionLabel{Notenbeispiel}
\setXmpListName{Verzeichnis der Notenbeispiele}
\end{verbatim}
\end{quote}

The \texttt{newfloat} package that was used to define the new floating environment offers a nice feature:
If you have loaded certain other packages it will automagically create some other commands of floating environments for you.
They behave the same as their standard counterparts, and for further information please refer to the respective package documentation.

\begin{description}
\item[\cmd{wrapmusicexample}] is created by loading the \texttt{wrapfig} package.
It will create a floating music example that is wrapped by the continuous text.
\item[\cmd{sidewaysmusicexample}] is created by loading the \texttt{rotating} package.
It will create a music example that is rotated so you can use examples with a landscape page layout.
\item[\cmd{SCmusicexample}] is created by loading the \texttt{sidecap} package.
It will create a music example with the caption placed beside.
\item[{\cmd{FPmusicexample}}] is created by loading the \texttt{fltpage} package.
It will create a music example with the caption placed on the previous or next page.
As \texttt{musicexamples} offers a solution for full-page music examples (see below) you'll probably won't need this.
\end{description}

There are several commands that are intended to actually print music examples within these two environments -- although you can of course print any images inside (as in the first example).
\cmd{musicSFE} and \cmd{lilypondSFE} are used to print Single File Examples, which usually consist of one single music system.
Technically they consist of one \emph{file}, but it is recommended not to mix these concepts arbitrarily.
As with other commands there are “music” and “lilypond” versions of it, which  you should use to distinguish between music examples provided by arbitrary image files and examples generated by LilyPond.
From \LaTeX's point of view they are the same (the “lilypond” version calls the other one internally), but a Python script can use them to keep track of the LilyPond generated examples.
\todo{Note: The script hasn't been implemented yet.}

You use both commands by providing the file basename as the single argument, including the relative path but excluding the extension.
Internally these commands include the example with \cmd{includegraphics}, and you can pass an optional argument with the same options that you can use to include images normally.
Usually it is recommendable to prepare the images in advance and include them unaltered in order to get a consistent layout.

\todo{It is intended to implement a conditional resizing that scales the image only if it exceeds the text height or width.}

\begin{quote}
\begin{verbatim}
\lilypondSFE[scale=1.2]{examples/lilypond/example1}
\end{verbatim}
\end{quote}

\cmd{lilypondMFE} is used to print a Multi File Example generated by LilyPond.
There is no “music“ variant of this command because it expects a series of music systems in a form that is typical for LilyPond's \texttt{lilypond-book} style output.
(We will discuss some aspects on how to streamline the creation of these examples on the LilyPond side later in \fref{sec:lilypond-configuration}).
If you have a music examples consisting of several files you can still use \cmd{musicSFE} multiple times to print them manually.

To use \cmd{lilypondMFE} you provide it the file basename as the mandatory argument, just like with the other commands we saw so far.
To work correctly the command expects a set of files to be present at the location specified by the argument:
\begin{description}
\item[\texttt{BASENAME-\#(.pdf)}] Numbered files for each system.
The music systems start with \#\,3, while the first two files are the book and score title markups, which are discarded by the command.
\item[\texttt{BASENAME-systems.count}] is a file containing exclusively one number indicating the number of systems to be processed.
\cmd{lilypondMFE} will iterate over the files 3 to n to consecutively print the systems using the \cmd{musicSFE} command.
If you need to have titles you will have to supply them directly within the music example environment.
\end{description}

\begin{itemize}
\item \texttt{musicexample}
\item \texttt{musicexampleNonFloat}
\item \cmd{onePageMusicExample}
\item \cmd{onePageLilyPondExample}
\item \cmd{multiPageMusicExample}
\item \cmd{multiPageLilyPondExample}
\end{itemize}

\section{The LilyPond Configuration Files}
\label{sec:lilypond-configuration}

\section{The Python Scripts}

\chapter{Implementation}

\section{musicexamples.sty}

\section{LilyPond Files}

\chapter{Licenses}
\input{licenses.inp}

\end{document}